As can be seen in table \ref{tab:key}, the normalized RGB color space shows the best results for automated image classification in comparison with the other color spaces when key-point SIFT is used. This result can be contributed to the fact that this color space is scale-invariant and invariant to light intensity changes. The opponent color space is also shift-invariant with respect to light intensity and shows a similar, but slightly lower, performance as the normalized RGB color space.  As the images of the dataset are mostly captured from the same point of view and have similar scales, the scale-invariance feature of the normalized RGB only slightly contributes to the automated image classification. The gray and RGB color spaces do not exhibit invariant features and therefore perform worse than the opponent and normalized RGB color spaces. These results are consistent with results from previous research by Sande et al.\cite{van2010evaluating}

\begin{table}[H]
\begin{tabular}{|c|ccccc|}
\hline
\textbf{Color space} & \textbf{AP Airplanes} & \textbf{AP Cars} & \textbf{AP Faces} & \textbf{AP Motorbikes} & \textbf{MAP}\\
\hline
gray & 0.8564 & 0.7468 & 0.6014 & 0.8485 & 0.7633\\
normalizedRGB & 0.9245 & 0.8081 & 0.8906 & 0.8308 & 0.8635 \\
RGB & 0.8870 & 0.8407 & 0.6821 & 0.8815 & 0.8228 \\
opponent & 0.8918 & 0.7946 & 0.8918 & 0.8457 & 0.8560\\
\hline
\end{tabular}
\caption{Effect of different color spaces for key SIFT}
\label{tab:key}
\end{table}

Table \ref{tab:color_sift} displays the results of different color spaces in use for dense SIFT. In contrast to the key-point SIFT method, the opponent color space gives the highest Mean Average Precision score for the dense SIFT method.  Surprisingly, normalized RGB color space obtains the lowest Mean Average Precision score for dense SIFT and is in contrast with the results of normalized RGB color for key-point SIFT. Because the default parameters for the window sizes are set small (4 and 8 pixels), features are extracted from small areas and more local features are extracted. The normalization of RGB possibly discards more distinctive local features than the other color space SIFT methods and therefore performs worse.\\

\begin{table}[H]
\begin{tabular}{|c|ccccc|}
\hline
\textbf{Color space} & \textbf{AP Airplanes} & \textbf{AP Cars} & \textbf{AP Faces} & \textbf{AP Motorbikes} & \textbf{MAP}\\
\hline
gray & 0.6476 & 0.7741 & 0.4292 & 0.8165 & 0.6668\\
normalizedRGB & 0.5743 & 0.3402 & 0.4678 & 0.5646 & 0.4867\\
RGB & 0.6424 & 0.9726 &  0.8689 & 0.6747 & 0.7897\\
opponent & 0.6447 & 0.9053 & 0.9510 & 0.7516 & 0.8132\\
\hline
\end{tabular}
\caption{Effect of different color spaces for dense SIFT}
\label{tab:color_sift}
\end{table}

