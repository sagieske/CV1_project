\subsubsection{Key-point SIFT}
The results for different color spaces using key-point SIFT are displayed in table \ref{tab:key}. Normalized RGB outperforms other color spaces when using key-point SIFT. This might be because Normalized RGB is scale-invariant and invariant to light intensity changes. The opponent color space is also shift-invariant with respect to light intensity and shows a similar, but slightly lower, performance compared to the normalized RGB color space. Since the images in the dataset are mostly captured from the same angle and are similarly scaled, the scale-invariance feature of the normalized RGB only slightly contributes to the automated image classification. The gray and RGB color spaces do not exhibit invariant features and therefore perform worse than the opponent and normalized RGB color spaces. These results are consistent with earlier research\cite{van2010evaluating}.

\begin{table}[H]
\begin{tabular}{|c|ccccc|}
\hline
\textbf{Color space} & \textbf{AP Airplanes} & \textbf{AP Cars} & \textbf{AP Faces} & \textbf{AP Motorbikes} & \textbf{MAP}\\
\hline
gray & 0.8564 & 0.7468 & 0.6014 & 0.8485 & 0.7633\\
normalizedRGB & 0.9245 & 0.8081 & 0.8906 & 0.8308 & 0.8635 \\
RGB & 0.8870 & 0.8407 & 0.6821 & 0.8815 & 0.8228 \\
opponent & 0.8918 & 0.7946 & 0.8918 & 0.8457 & 0.8560\\
\hline
\end{tabular}
\caption{Effect of different color spaces for key SIFT}
\label{tab:key}
\end{table}

\subsubsection{Dense SIFT}
Table \ref{tab:color_sift} displays the effect of different color spaces for dense SIFT. In contrast to the key-point SIFT method, for dense sampling, the highest Mean Average Precision is found using the opponent color space.  Surprisingly, normalized RGB color space obtains the lowest Mean Average Precision score for dense SIFT, which is in contrast with the results of normalized RGB color for key-point SIFT. To explain this, one should consider that the default parameters for the window sizes are small (4 and 8 pixels) and thus features are extracted from small areas, resulting in many local instead of few global features. The normalization of RGB possibly discards more distinctive local features than the other color space SIFT methods and therefore performs worse.\\

\begin{table}[H]
\begin{tabular}{|c|ccccc|}
\hline
\textbf{Color space} & \textbf{AP Airplanes} & \textbf{AP Cars} & \textbf{AP Faces} & \textbf{AP Motorbikes} & \textbf{MAP}\\
\hline
gray & 0.6476 & 0.7741 & 0.4292 & 0.8165 & 0.6668\\
normalizedRGB & 0.5743 & 0.3402 & 0.4678 & 0.5646 & 0.4867\\
RGB & 0.6424 & 0.9726 &  0.8689 & 0.6747 & 0.7897\\
opponent & 0.6447 & 0.9053 & 0.9510 & 0.7516 & 0.8132\\
\hline
\end{tabular}
\caption{Effect of different color spaces for dense SIFT}
\label{tab:color_sift}
\end{table}

