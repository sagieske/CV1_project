\documentclass{article} % For LaTeX2e
\usepackage{times}
\usepackage{hyperref}
\usepackage{url}
\usepackage{graphicx}
\usepackage{cite}
\usepackage{times}
\usepackage{tikz}
\usepackage[]{algorithm2e}
\usetikzlibrary{arrows,decorations.pathmorphing,fit,positioning}
\usepackage{algorithmic}
\usepackage{mdframed}
\usepackage[margin=1in]{geometry}
\usepackage{amsmath}
\usepackage{graphicx}
\usepackage{caption}
\usepackage{subcaption}
\usepackage{float}
%\documentstyle[nips14submit_09,times,art10]{article} % For LaTeX 2.09


\title{Image classification using bag-of-words model}


\author{
Sharon Gieske \\
\texttt{6167667}\\
\texttt{sharongieske@gmail.com} \\
\and
David van Erkelens\\
\texttt{10264019}\\
\texttt{daviddvanerkelens@gmail.com} \\
\and
Elise Koster \\
\texttt{5982448}\\
\texttt{koster.elise@gmail.com}
}


% The \author macro works with any number of authors. There are two commands
% used to separate the names and addresses of multiple authors: \And and \AND.
%
% Using \And between authors leaves it to \LaTeX{} to determine where to break
% the lines. Using \AND forces a linebreak at that point. So, if \LaTeX{}
% puts 3 of 4 authors names on the first line, and the last on the second
% line, try using \AND instead of \And before the third author name.

\newcommand{\fix}{\marginpar{FIX}}
\newcommand{\new}{\marginpar{NEW}}

%\nipsfinalcopy % Uncomment for camera-ready version
\newcommand{\specialcell}[2][c]{%
  \begin{tabular}[#1]{@{}c@{}}#2\end{tabular}}
\begin{document}


\maketitle

\begin{abstract}
A bag-of-words approach to image classification was used  to classify four different image categories. After exploring the parameter space, the best model reports a mean average precision of over 97\%. In this instance, the best model used dense sampling in an opponent color space for feature extraction, and classified using a linear kernel.
\end{abstract}


\section{Introduction}
Object classification is a fundamental part of Computer Vision and can be used to automate processes and provide environmental reasoning for robotics. Consequently, a high accuracy in object classification systems is invaluable for industry and science alike.\\
This paper outlines the results of a project analyzing a bag-of-words approach to image classification, and explores the effectiveness of different parameter settings.\\
The data section will describe the images used for training and testing, the implementation section will introduce the techniques used and the results section will describe the difference in performance for each set of techniques. The conclusion will report on the optimal combination found and the section future work will comment on possible improvements.

\section{Data}
The training data consists of 2000 .jpg-images in four classes (500 per class): airplanes, cars, faces and motorbikes. The group of training images is split into a vocabulary training set and a classification training set.The test data consists of 200 .jpg-images in the same four classes, all of whom are used for testing.

\section{Implementation}
Instead of training classifiers on a large set of pixels, the bag-of-words approach is used. This approach first extracts features from images and subsequently uses them to build a vocabulary of visual `words'. Each image can then be described as a set of these words, which makes training a classifier easier and faster than a pixel-by-pixel approach. \\
\subsection{SIFT}
To be able to build a bag-of-words, features need to be extracted from each training image. This is done using Scale Invariant Feature Transform (SIFT), an algorithm that detects points of interest in an image and produces descriptors of these features. Two types of SIFT are used: key-point (produces descriptors of points of interest) and dense-sampling (for every $n$ pixels a descriptor is produced). A multitude of color spaces is used: gray-scale, RGB (regular .jpg-image with three channels), normalized RGB (normalizedRGB) and opponent, where $R,G,B$ respectively are the pixel values per color channel, and
\begin{align*}
\text{For normalizedRGB:}\\
r&= \frac{R}{R+G+B}\\
g&=\frac{G}{R+G+B} \\
b&= \frac{B}{R+G+B}\\
\text{For opponent:}\\
O_1&= \frac{R-G}{\sqrt{2}}\\
O_2&= \frac{R+G-2B}{\sqrt{6}}\\
O_3&= \frac{R+G+B}{\sqrt{3}}
\end{align*}
Each different color space defines different intensities for each pixel in a color channel, and thus causes SIFT to return different descriptors.
\subsection{K-means}
Performing SIFT on the first set of training images results in a set of descriptors for each image, which are clustered into visual words using K-means. The K-means algorithm works by calculating the Euclidian distance between each data point and a cluster-center (the mean), and iteratively re-calculating the means and re-assigning the data-points until convergence. The resulting clusters form the visual vocabulary used for describing and classifying images later on. 
\subsection{SVM classification}
Once the visual vocabulary has been built, features are extracted from a new set of training images. These features are grouped into words according to the visual vocabulary, and for each image a histogram of visual word frequencies is computed.
These histograms are used as input to train four Support Vector Machines (SVMs) (one per class), using different kernel-functions. SVMs are non-parametric classifiers, which work by maximizing the margin between the decision boundary and two classes of data.
After training, all test images are classified according to the SVM-models built using the training images. 
\subsection{Evaluation}
The classification systems are evaluated using the Average Precision (AP) for each class and Mean Average Precision (MAP) over all classes. The Average Precision for a single class is defined as
\begin{align}
\frac{1}{m_c} \sum\limits_{i=1}^N\frac{f_c(x_i)}{i}
\end{align}
where $N$ is the total number of images, $m_c$ is the number of images of class $c$, $x_i$ is the ith image in the ranked list and $f_c$ is a function which returns the number of images of class $c$ in the first $i$ images if $x_i$ is of class $c$, and 0 otherwise.
\newpage
\section{Supplied files}
The following files have been supplied:\\

\begin{itemize}
\item \verb|frame.m|\\
The framework that runs all functions in order.
\item \verb|demo.m|\\
A demo function that runs the code with default values for parameters.
\item \verb|descriptors_all_classes.m|\\
Used in training to create descriptors for all classes.
\item \verb|get_descriptors_class.m|\\
Called by \verb|descriptors_all_classes.m| to create descriptors for a single class.
\item \verb|extract_features3.m|\\
Extracts features from a single image.
\item \verb|get_histogram.m|\\
Builds a histogram of visual words.\\
\item \verb|quantize_features.m|
Creates a list with counts for visual words.
\item \verb|testing.m|\\
Classifies test images and evaluates the model.
\end{itemize}

\section{Results}
This section contains an overview and short analysis of the different results obtained during this project. \\
The default settings for the experiments run in this research are shown in table \ref{tab:default}.

\begin{table}[H]
\begin{tabular}{|l|l|}
\hline
\textbf{Parameter} & \textbf{Default value}\\
\hline
Color space & Opponent\\
SIFT type & Dense\\
Training set size feature histogram & 75 per class\\
Training set size SVM & 75 per class\\
Test set size & 50 per class \\
Cluster size & 400 \\
Step size (only for dense SIFT) & 20\\
\hline
\end{tabular}
\caption{Default values for experiments}
\label{tab:default}
\end{table}

\subsection{Effect of different color spaces in key-point SIFT}

\begin{table}[H]
\begin{tabular}{|c|ccccc|}
\hline
\textbf{Color space} & \textbf{Airplanes} & \textbf{Cars} & \textbf{Faces} & \textbf{Motorbikes} & \textbf{Average}\\
\hline
gray & & & & & \\
normalizedRGB & & & & & \\
RGB & & & & & \\
opponent & & & & & \\
\hline
\end{tabular}
\caption{Effect of different color spaces for key SIFT}
\end{table}


\subsection{Effect of different color spaces in dense SIFT}

\begin{table}[H]
\begin{tabular}{|c|ccccc|}
\hline
\textbf{Color space} & \textbf{Airplanes} & \textbf{Cars} & \textbf{Faces} & \textbf{Motorbikes} & \textbf{Average}\\
\hline
gray & & & & & \\
normalizedRGB & & & & & \\
RGB & & & & & \\
opponent & 0.6447 & 0.9053 & 0.9510 & 0.7516 & 0.8132\\
\hline
\end{tabular}
\caption{Effect of different color spaces for dense SIFT}
\end{table}



\subsection{Effect number of training samples for feature histogram}

\begin{table}[H]
\begin{tabular}{|c|ccccc|}
\hline
\textbf{Training samples} & \textbf{Airplanes} & \textbf{Cars} & \textbf{Faces} & \textbf{Motorbikes} & \textbf{Average}\\
\hline
30 & 0.1394 & 0.3118& 0.1924& 0.1394 & 0.1958\\
50 & 0.1394 & 0.3118& 0.1924& 0.1394 & 0.1958\\
70 & 0.6563 & 0.9534 & 0.9191 & 0.7830 & 0.8280\\
75 & 0.6447 & 0.9053 & 0.9510 & 0.7516 & 0.8132\\
90 & 0.6749 & 0.9157 & 0.9631 & 0.7798 & 0.8334\\
\hline
\end{tabular}
\caption{Effect number of training samples (per class) for feature histogram, Sift type: dense, Color space: opponent}
\end{table}


\subsection{Effect number of training samples for SVM}

\begin{table}[H]
\begin{tabular}{|c|ccccc|}
\hline
\textbf{Training samples} & \textbf{Airplanes} & \textbf{Cars} & \textbf{Faces} & \textbf{Motorbikes} & \textbf{Average}\\
\hline
30 & & & & & \\
50 & & & & & \\
70 & & & & & \\
75 & 0.6447 & 0.9053 & 0.9510 & 0.7516 & 0.8132\\
90 & & & & & \\
\hline
\end{tabular}
\caption{Effect number of training samples (per class) for SVM, Sift type: dense, Color space: opponent}
\end{table}


\subsection{Effect of different cluster sizes}

\begin{table}[H]
\begin{tabular}{|c|ccccc|}
\hline
\textbf{Cluster size} & \textbf{Airplanes} & \textbf{Cars} & \textbf{Faces} & \textbf{Motorbikes} & \textbf{Average}\\
\hline
400 & 0.6447 & 0.9053 & 0.9510 & 0.7516 & 0.8132\\
800 & 0.6424 & 0.5067 & 0.9602 & 0.5667 & 0.6690\\
1600 & 0.6438 & 0.3115 & 0.8410 & 0.5361 & 0.5831 \\
2000 & 0.6367 & 0.7253 & 0.3357 & 0.1926 & 0.4726\\
4000 & & & & & \\
\hline
\end{tabular}
\caption{Effect of different cluster sizes, Sift type: dense, Color space: opponent}
\end{table}

\subsection{Effect of different step sizes for dense}

\begin{table}[H]
\begin{tabular}{|c|ccccc|}
\hline
\textbf{Step size} & \textbf{Airplanes} & \textbf{Cars} & \textbf{Faces} & \textbf{Motorbikes} & \textbf{Average}\\
\hline
10 & 0.6823 & 0.8654 & 0.9444 & 0.6934 & 0.7964\\
20 & 0.6447 & 0.9053 & 0.9510 & 0.7516 & 0.8132\\
30 & & & & & \\
50 & & & & & \\
\hline
\end{tabular}
\caption{Effect of different step sizes for dense, Sift type: dense, Color space: opponent}
\end{table}


\subsection{Effect different kernels for SVM}

\begin{table}[H]
\begin{tabular}{|c|ccccc|}
\hline
\textbf{Kernel} & \textbf{Airplanes} & \textbf{Cars} & \textbf{Faces} & \textbf{Motorbikes} & \textbf{Average}\\
\hline
sigmoid & 0.6447 & 0.9053 & 0.9510 & 0.7516 & 0.8132\\
linear & & & & & \\
polynomial & & & & & \\
radial & & & & & \\
\hline
\end{tabular}
\caption{Effect different kernels for SVM, Sift type: dense, Color space: opponent}
\end{table}


\section{Conclusion}
The bag-of-words approach to image classification as explored in this report shows promising results, reporting a mean average precision over four classes of over 97\%. Manual inspection of the highest ranked data for different classifiers showed a dependence on context cues (e.g. background color), though these context cues did not overrule the actual object photographed. However, the images used for training and testing were singular, showing objects from specific angles. Therefore, it cannot be conclusively stated that these results will be replicable in studies with a more varied dataset. Still, it is believed that this approach will work quite well for more varied data, especially if features are extracted using SIFT with a shift-invariant colorspace (opponent), since SIFT is invariant to translation, scaling and rotation. \\
Dense sampling outperforms key-point sampling, implying that even non-key point features (possibly background/context features) are relevant to the type of images. For dense sampling, the best step size was small, allowing the program to pick up local changes, but not too small, preventing the program from fussing over unnecessary details. Moreover, the use of a range ($>2$ types) of window sizes that were neither overly small ($<$8 pixels) nor very large ($>32$ pixels) has shown increase in performance. \\
The best color space for feature extraction was found to be opponent, as it is shift invariant, whereas normalizedRGB greatly underperforms, possibly because the normalization removes important context-based feature information. \\
Different codebook sizes for building feature histograms have been tested, and in conclusion, large codebook sizes result in overfitting, as the model starts fitting to noise. Overly small codebook sizes underfit, however. The best codebook size lies between these points, and is dependent on the dataset. Therefore, this threshold should probably be experimentally determined. \\
For the dataset at hand, the best performance was found for an SVM with a linear kernel, which is less complex than other tested kernels and therefore less prone to overfitting. \\
In conclusion, the bag-of-words approach performs quite well, but should be explored with different datasets before any final conclusions can be drawn.

\section{Future work}
It is hypothesized that for dense SIFT, different step sizes will work better with different window sizes. For example, in small window sizes, the step sizes need to be smaller to allow for the capturing of all local features, whereas for larger window sizes, these steps can be larger while still capturing all required features. \\
Another suggestion for future research is applying the bag-of-words model to more complex datasets, where objects are photographed from different angles, partly occluded, etcetera. Using the model on a more real-life dataset will allow for conclusions to be drawn about its usefulness.
\bibliographystyle{plain}
\bibliography{bibliography}

\end{document}
